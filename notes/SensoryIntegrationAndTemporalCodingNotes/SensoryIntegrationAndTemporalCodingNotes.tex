\documentclass[11pt, a4paper, oneside]{article}   	% use "amsart" instead of "article" for AMSLaTeX format
\usepackage[margin=1in]{geometry}                		% See geometry.pdf to learn the layout options. There are lots.
%\usepackage{fullpage}
\usepackage{enumitem}
\usepackage{notetaking}
\geometry{letterpaper}                   		% ... or a4paper or a5paper or ... 
%\geometry{landscape}                		% Activate for rotated page geometry
%\usepackage[parfill]{parskip}    		% Activate to begin paragraphs with an empty line rather than an indent
\usepackage{graphicx}				% Use pdf, png, jpg, or eps§ with pdflatex; use eps in DVI mode
								% TeX will automatically convert eps --> pdf in pdflatex		
\usepackage{hyperref}
\usepackage{xcolor}
\usepackage{amssymb}

\usepackage{mathtools} % look into this for math tools needs!

%SetFonts

%SetFonts

%\setlist[enumerate]{label*=\arabic*.} % this allows enumerations to be nested and have a label like 1.3.2
\setlist[enumerate,2]{label=\roman*)}
\setlist[enumerate,3]{label=\alph*)}
\setlist[enumerate,4]{label=\Alph*)}



\renewcommand{\outlinei}{enumerate}
\renewcommand{\outlineii}{enumerate}
\renewcommand{\outlineiii}{enumerate}
\renewcommand{\outlineiiii}{enumerate}

\usepackage{tikz}
\def\checkmark{\tikz\fill[scale=0.4](0,.35) -- (.25,0) -- (1,.7) -- (.25,.15) -- cycle;}



\hypersetup{
  colorlinks   = true,
  citecolor    = gray
}
\title{Sensory Integration and Temporal Coding Notes}
\author{Andrew McDonald}
%\date{}							% Activate to display a given date or no date
\begin{document}
\maketitle
\hypersetup{linkcolor=blue}
\tableofcontents
\hypersetup{linkcolor=blue}
\listoffigures
\hypersetup{linkcolor=blue} %red

\section{Ortus: Moving Forward}

\section{Thoughts}

\begin{outline}
    \point point\ldots
    
\end{outline}








\section{Multimodal sensory integration in single cerebellar granule cells in vivo. \cite{Ishikawa2015}}

Look up: granule cells, mossy fibers, synaptic facilitation, postsynaptic inhibition via golgi cells

\begin{outline}
    \point Brain integrates multimodal sensory signals (midbrain, cerebral association cortex, and cerebellum). 
    \point It is well known that each granule cell in the mammalian cerebellum receives excitatory synaptic inputs from an avg. of four mossy fibers
    \point Patch--clamp recordings from single cerebellar granule cells of rats in vivo examined their responses to auditory, visual and somatosensory stimulation.
    \point Individual granule cells were found that responded to stimulation of three separate sensory modalities (Fig. 2)
        \subpoint see fig 2
    \point Focusing on the first EPSC (excitatory postsynaptic current) burst resulting from a sensory stimulation, the amplitude of auditory--evoked EPSC's (~16 pA) was significantly larger than that of somatosensory--evoked EPSC's (~8 pA), indicating the two groups of EPSCs originated from different synapses (so, conveyed by different mossy fibers). 
        \subpoint The second and subsequent events are likley to be affeted by synaptic facilitation and depression
    \point Across the population, 8/20 of multisensory cells showed significantly different first EPSC amplitudes in response to different modalities. One cell showed significant different in EPSC rise time for the different modalities.
    \subpoint NOTE: Could different activation level rise times indicate something? Perhaps different types of information are relayed by different neurons; e.g., one group combines the input to say \textit{what} it is, and another just says how \textit{strong} it is. Basically, group A differentiates between visual, touch, smell, or auditory, while group B says ``this is the intensity of whatever stimulus you're getting''. Obviously there are some issues with this, like color vision, hearing frequencies, etc., but it could play a role.
    \point Combined stimulation of two modalities evoked responses that were approximately the sum of two unimodal responses, though the summation was typically sub--linear.
    \point 2/4 granule cells showed a greater number of APs when combining two sensory modalities as opposed to two unimodal stimuli.
    \point Their work shows that multisensory signals converge onto sindividual granule cells in vivo, and that multisensory input can enhance granule cell spike output
    \point As granule cells receive excitatory input from only 4 mossy fibers (on average), it's possible the 4 sensory modalities can be passed into granule cells from 4 mossy fibers.
    \point NOTE: In Fig 1 A, you can see that the auditory stimulus well outlasts most of the measured responses all of which have initial spikes right at the onset of the stimulus (3 of the four neurons have one spike, and the 3rd has 3, with the first two close together, and the third about 1/3 of the magnitude, and a little over halfway through the auditory stimulus. This suggests that there might be different pathways that are responsible for delivering different types of information, as I suggested above. (So, they may not have measured the neurons that were responsible for relaying the duration of the stimulus).
    
\end{outline}

\section{Rate and Temporal Coding Convey Multisensory Information in Primary Sensory Cortices \cite{Bieler2017}}

\begin{outline}
    \point
    
\end{outline}

\section{Multisensory Interactions Influence Neuronal Spike Train Dynamics in the Posterior Parietal Cortex \cite{VanGilder2016}}

\section{Synaptic diversity enables temporal coding of coincident multisensory inputs in single neurons. \cite{Chabrol2015}}
\section{Sensorimotor Representations in Cerebellar Granule Cells in Larval Zebrafish Are Dense, Spatially Organized, and Non-temporally Patterned \cite{Knogler2017}}

\section{Rate coding versus temporal order coding: what the retinal ganglion cells tell the visual cortex. \cite{VanRullen2001}}

\section{Multisensory integration and cross-modal learning in synaesthesia: A unifying model \cite{Newell2016}}

\section{Optimal Degrees of Synaptic Connectivity \cite{Litwin-Kumar2017}}

\section{Pyramidal neurons: dendritic structure and synaptic integration. \cite{Spruston2008}}

\section*{\centering{\textbf{Papers from previous notes file \textit{below}!}}}
\section{Timing Rules for Synaptic Plasticity Matched to Behavioral Function \cite{Suvrathan2016}}
\label{MNCF:Suvrathan2016}

from \url{http://www.neuroanatomy.wisc.edu/cere/text/P4/climb.htm}
A single action potential from a climbing fiber elicits a burst of action potentials in the Purkinje Cells that it contacts. This burst is called a complex spike. Climbing fibers are ``lazy'' (but strong), thus Purkinje cells exhibit complex spikes at a rate of about 1 per second. 

going to come back to this paper at some point\ldots



\begin{outline}
    \point Synaptic plasticity rules themselves can be highly specialized to match the functional requirements of a learning task
    \point The fundamental requirement of associative learning is to store information about the correlations between events
        \subpoint synaptic plasticity mechanisms have been described that can capture the correlations between coincident, or nearly coincident events
           \subsubpoint \textbf{Feldman, D.E. (2012). The spike-timing dependence of plasticity. Neuron 75, 556–571.}
        \subpoint Behavioral observations indicate the brain is also able to associate events separated in time, with requisite temporal precision
            \subsubpoint During feedback--based learning, a delayed error signal must selectively modify synapses active at the specific, earlier time when the neural command leading to an error was generated
            \supersubpoint known as the ``temporal credit assignment'' problem -- think of feedback delay when throwing a ball
    \point During cerebellum--dependent learning, delayed feedback about performance errors is conveyed to the cerebellum by its climbing fiber input.
        \subpoint Each spike in a climbing fiber produces a ``complex spike'', and concomitant calcium influx in its Purkinje cell targets. Related pairings of climbing fiber (CF) activation with the activation of parallel fiber (PF) synapses onto the Pukinje cells result in depression of the parallel fiber--to--Purkinje cell (PF--to--PC) synapses.
            \subsubpoint Thus, error signals carried by the climbing fibers are thought to sculpt away, through associate synaptic depression, PF--to--PC synapses that were active around the time that an error was generated

    
\end{outline}






%\section{Neural plasticity and behavior ??? sixty years of conceptual advances \cite{Sweatt2016}}


\section{Homeostatic Plasticity of Subcellular Neuronal Structures: From Inputs to Outputs \cite{Wefelmeyer2016}}

Coming soon\ldots

\section{Mind the Gap Junctions: The Importance of Electrical Synapses to Visual Processing \cite{Demb2016}}

Coming soon\ldots

\section{Relational associative learning induces cross-modal plasticity in early visual cortex \cite{Headley2015}}

Coming soon\ldots

\section{Neuroscience: When perceptual learning occurs \cite{Sasaki2017}}

``A study now finds that visual perceptual learning of complex features occurs due to enhancement of later, decision-related stages of visual processing, rather than earlier, visual encoding stages. It is suggested that strengthening of the readout of sensory information between stages may be reinforced by an implicit reward learning mechanism.''


Note: Just glanced at paper, but this suggests that the approach I want to take with the visual system, having groups of neurons cluster together in effect, (described above), may be exactly what is happening in the brain.


%\section{Chapter 13 - Neural Circuit Mechanisms of Value-Based Decision-Making and Reinforcement Learning \cite{Soltani2017}}
%\section{Structure of plasticity in human sensory and motor networks due to perceptual learning. \cite{Vahdat2014}}
%
%Note: Just read ``Discussion'' section---paper wasn't exactly looking at what I thought it would, but the discussion section seems to suggest that the idea of expanding simple sub--networks (e.g., go from one ``touch'' neuron to a group as the need arises to differentiate between multiple types of touches) makes sense:
%
%\begin{outline}
%   \point Perceptual training induces plasticity in the motor system that cannot be explained by activity in the somatosensory network
%   \point Perceptual training changes the characteristics of subpsequent movements, and to improe somatosensory perceptual judgements
%   \point 
%\end{outline}



\section{Why Neurons Have Thousands of Synapses, a Theory of Sequence Memory in Neocortex \cite{Hawkins2016}}

Coming soon\ldots

\section{Integrating Hebbian and homeostatic plasticity : introduction \cite{Fox2017}}


Coming soon\ldots

\section{Homeostatic plasticity mechanisms in mouse V1 \cite{Kaneko2017}}

NEXT AS WELL PAPER 

Coming soon\ldots

\section{Synaptic scaling rule preserves excitatory–inhibitory balance and salient neuronal network dynamics \cite{Barral2016}}

NEXT NEXT NEXT NEXT PAPER

Coming Soon\ldots


\section{Rapid Encoding of New Memories by Individual Neurons in the Human Brain \cite{Ison2015}}

\begin{outline}
    \point individual neurons were measured upon showing subjects a picture of a family member, the Eiffel tower, and both together in order to attempt to create an association. The family member alone caused a mean firing rate of 13.1 spikes/s, the Eiffel tower along caused a mean firing rate of 3.6 spikes/s, After a single exposure to the composite picture, the mean response to the Eiffel tower alone rose to 7.6 spikes/s. The study did tests to ensure that the change in firing rate was a result of the association rather than familiarity.  
    \point The authors suggest this is a result of single--cell encoding, however, to me:
        \subpoint NOTE: This suggests that there is an immediate change in the structure of the brain, that occurs as a result of new associations forming
        \subpoint NOTE: Further, as it was individual neurons that were being measured, there must be a chain of reactions (or numerous parallel reactions) occurring, that cause formation and strengthening of synapses (otherwise, it seems highly unlikely that the study happened to measure the one neuron that would undergo changes).
            \subsubpoint NOTE: this suggests a global rule\ldots
\point ``repetition suppression'' is a neural mechanism that gradually decreases the intensity of response to a repeated stimulus
    \subpoint can have repetition suppression with plasticity; which might be why it's better to learn things over a long period of time rather than all at once. due to repetition suppression, the stimulus decreases in intensity, so the strengthening of new synapses slows. But, if you do it over time, the repetition suppression decays, and you get the full strength of the stimulus back.
\end{outline}


\section{Functional and structural underpinnings of neuronal assembly formation in learning \cite{Holtmaat2016}}

NEXT NEXT PAPER

Coming Soon\ldots


\section{Mirror Neurons from Associative Learning \cite{Catmur2016}}

Coming Soon\ldots


\section{Associative learning and sensory neuroplasticity: how does it happen and what is it good for? \cite{McGann2015}}

NEXT NEXT NEXT PAPER
 
Coming Soon\ldots


\section{Associative learning rapidly establishes neuronal representations of upcoming behavioral choices in crows. \cite{Veit2015}}

Coming Soon\ldots


\section{Associative Learning Drives the Formation of Silent Synapses in Neuronal Ensembles of the Nucleus Accumbens \cite{Whitaker2015}}

Coming Soon\ldots

\section{The neurobiology shaping affective touch: Expectation, motivation, and meaning in the multisensory context \cite{Ellingsen2016}}

Coming Soon\ldots

\bibliography{SensoryIntegrationAndTemporalCoding,ShortenedBibFromCircuitDevelopmentNotes}
\bibliographystyle{ieeetr}
\end{document}  

%%%%%%%%%%%% 
% There is a function, callable via ":call BibToSections()" in ~./vimrc that will run all these commands in successsion:

%% command to delete .bib stuff except for @<thing> and title lines:
% :g /^\(\<\(author\|doi\|file\|abstract\|publisher\|journal\|pages\|url\|year\|volume\|pmid\|keywords\|issn\|isbn\|booktitle\|number\|edition\)\>\|}\)\{1}.*$/ d 

%% command to delete characters around reference 'name':
% :%s/@.\{-}{\(.*\),/\point/g

%% command to replace "}}," from the title line, with \cite{ and the reference name from the @... line:
% :%s/}},/\=" \\cite{".getline(line('.')-1)."}}"/g    

%% command to replace "title = {{" with "\section{"
% :%s/title\ =\ {{\(.*}}\)/\\section{\point/g     

%% and finally, a command to delete all lines that don't begin with \section
% :g!/\\section.*/d 

